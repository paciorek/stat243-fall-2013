\batchmode
\makeatletter
\def\input@path{{/accounts/gen/vis/paciorek/teaching/243/stat243-fall-2013/lectures//}}
\makeatother
\documentclass[12pt]{article}\usepackage{graphicx, color}
%% maxwidth is the original width if it is less than linewidth
%% otherwise use linewidth (to make sure the graphics do not exceed the margin)
\makeatletter
\def\maxwidth{ %
  \ifdim\Gin@nat@width>\linewidth
    \linewidth
  \else
    \Gin@nat@width
  \fi
}
\makeatother

\definecolor{fgcolor}{rgb}{0.2, 0.2, 0.2}
\newcommand{\hlnumber}[1]{\textcolor[rgb]{0,0,0}{#1}}%
\newcommand{\hlfunctioncall}[1]{\textcolor[rgb]{0.501960784313725,0,0.329411764705882}{\textbf{#1}}}%
\newcommand{\hlstring}[1]{\textcolor[rgb]{0.6,0.6,1}{#1}}%
\newcommand{\hlkeyword}[1]{\textcolor[rgb]{0,0,0}{\textbf{#1}}}%
\newcommand{\hlargument}[1]{\textcolor[rgb]{0.690196078431373,0.250980392156863,0.0196078431372549}{#1}}%
\newcommand{\hlcomment}[1]{\textcolor[rgb]{0.180392156862745,0.6,0.341176470588235}{#1}}%
\newcommand{\hlroxygencomment}[1]{\textcolor[rgb]{0.43921568627451,0.47843137254902,0.701960784313725}{#1}}%
\newcommand{\hlformalargs}[1]{\textcolor[rgb]{0.690196078431373,0.250980392156863,0.0196078431372549}{#1}}%
\newcommand{\hleqformalargs}[1]{\textcolor[rgb]{0.690196078431373,0.250980392156863,0.0196078431372549}{#1}}%
\newcommand{\hlassignement}[1]{\textcolor[rgb]{0,0,0}{\textbf{#1}}}%
\newcommand{\hlpackage}[1]{\textcolor[rgb]{0.588235294117647,0.709803921568627,0.145098039215686}{#1}}%
\newcommand{\hlslot}[1]{\textit{#1}}%
\newcommand{\hlsymbol}[1]{\textcolor[rgb]{0,0,0}{#1}}%
\newcommand{\hlprompt}[1]{\textcolor[rgb]{0.2,0.2,0.2}{#1}}%

\usepackage{framed}
\makeatletter
\newenvironment{kframe}{%
 \def\at@end@of@kframe{}%
 \ifinner\ifhmode%
  \def\at@end@of@kframe{\end{minipage}}%
  \begin{minipage}{\columnwidth}%
 \fi\fi%
 \def\FrameCommand##1{\hskip\@totalleftmargin \hskip-\fboxsep
 \colorbox{shadecolor}{##1}\hskip-\fboxsep
     % There is no \\@totalrightmargin, so:
     \hskip-\linewidth \hskip-\@totalleftmargin \hskip\columnwidth}%
 \MakeFramed {\advance\hsize-\width
   \@totalleftmargin\z@ \linewidth\hsize
   \@setminipage}}%
 {\par\unskip\endMakeFramed%
 \at@end@of@kframe}
\makeatother

\definecolor{shadecolor}{rgb}{.97, .97, .97}
\definecolor{messagecolor}{rgb}{0, 0, 0}
\definecolor{warningcolor}{rgb}{1, 0, 1}
\definecolor{errorcolor}{rgb}{1, 0, 0}
\newenvironment{knitrout}{}{} % an empty environment to be redefined in TeX

\usepackage{alltt}
\usepackage[T1]{fontenc}
\usepackage[latin9]{inputenc}
\usepackage[letterpaper]{geometry}
\geometry{verbose,tmargin=1in,bmargin=1in,lmargin=1in,rmargin=1in}
\usepackage{setspace}
\onehalfspacing
\usepackage[unicode=true]
 {hyperref}

\makeatletter
%%%%%%%%%%%%%%%%%%%%%%%%%%%%%% User specified LaTeX commands.
\usepackage{/accounts/gen/vis/paciorek/latex/paciorek-asa,times,graphics}
\input{/accounts/gen/vis/paciorek/latex/paciorekMacros}
%\renewcommand{\baselinestretch}{1.5}
\hypersetup{unicode=true, pdfusetitle,
bookmarks=true,bookmarksnumbered=false,bookmarksopen=false,
 breaklinks=false,pdfborder={0 0 1},backref=false,colorlinks=true,}

\makeatother
\IfFileExists{upquote.sty}{\usepackage{upquote}}{}

\begin{document}




\title{Reading and Writing to/from R}

\maketitle
References: 
\begin{itemize}
\item Adler
\item Chambers
\item \href{http://cran.r-project.org/doc/manuals/R-intro.html}{R intro manual}
on CRAN (R-intro).
\item Venables and Ripley, Modern Applied Statistics with S
\item Murrell, Introduction to Data Technologies. 
\item \href{http://cran.r-project.org/doc/manuals/R-data.html}{R Data Import/Export manual}
on CRAN (R-data). 
\end{itemize}

\section{Data storage and formats (outside R)}

At this point, we're going to turn to reading data into R and manipulating
text, including regular expressions. We'll focus on doing these manipulations
in R, but the concepts involved in reading in data, database manipulations,
and regular expressions are common to other languages, so familarity
with these in R should allow you to pick up other tools more easily.
The main downside to working with datasets in R is that the entire
dataset resides in memory, so R is not so good for dealing with very
large datasets. More on alternatives in a bit. Another common frustration
is controlling how the variables are interpreted (numeric, character,
factor) when reading data into a data frame.

R has the capability to read in a wide variety of file formats. Let's
get a feel for some of the common ones. 
\begin{enumerate}
\item Flat text files (ASCII files): data are often provided as simple text
files. Often one has one record or observation per row and each column
or field is a different variable or type of information about the
record. Such files can either have a fixed number of characters in
each field (fixed width format) or a special character (a delimiter)
that separates the fields in each row. Common delimiters are tabs,
commas, one or more spaces, and the pipe (|). Common file extensions
are \emph{.txt} and \emph{.csv}. Metadata (information about the data)
is often stored in a separate file. I like CSV files but if you have
files where the data contain commas, other delimiters can be good.
Text can be put in quotes in CSV files. This was difficult to deal
with in the shell in PS1, but \emph{read.table()} in R handles this
situation. 

\begin{enumerate}
\item One occasionally tricky difficulty is as follows. If you have a text
file created in Windows, the line endings are coded differently than
in UNIX (a newline (the ASCII character\emph{ \textbackslash{}n})
and a carriage return (the ASCII character \emph{\textbackslash{}r})
in Windows vs. only a newline in UNIX). There are UNIX utilities (\emph{fromdos}
in Ubuntu, including the SCF Linux machines and \emph{dos2unix} in
other Linux distributions) that can do the necessary conversion. If
you see \emph{\textasciicircum{}M} at the end of the lines in a file,
that's the tool you need. Alternatively, if you open a UNIX file in
Windows, it may treat all the lines as a single line. You can fix
this with \emph{todos} or \emph{unix2dos}.\\
As a side note, Macs have line endings as in UNIX, but before Mac
OS X, lines ended only in a carriage return. There is a UNIX utility
call \emph{mac2unix} that can convert such text files. 
\end{enumerate}
\item In some contexts, such as textual data and bioinformatics data, the
data may in a text file with one piece of information per row, but
without meaningful columns/fields.
\item In scientific contexts, netCDF (\emph{.nc}) (and the related HDF5)
are popular format for gridded data that allows for highly-efficient
storage and contains the metadata within the file. The basic structure
of a netCDF file is that each variable is an array with multiple dimensions
(e.g., latitude, longitude, and time), and one can also extract the
values of and metadata about each dimension. The \emph{ncdf} package
in R nicely handles working with netCDF files. These are examples
of a binary format, which is not (easily) human readable but can be
more space-efficient and faster to work with (because they can allow
random access into the data rather than requiring sequential reading). 
\item Data may also be in the form of XML or HTML files. We won't deal with
these in 243, except to the extent that they come up in a problem
set.
\item Data may already be in a database or in the data storage of another
statistical package (\emph{Stata}, \emph{SAS}, \emph{SPSS}, etc.).
The \emph{foreign} package in R has excellent capabilities for importing
Stata (\emph{read.dta()}), SPSS (\emph{read.spss()}), SAS (\emph{read.ssd()}
and, for XPORT files, \emph{read.xport()}), and dbf (a common database
format) (\emph{read.dbf()}), among others.
\item For Excel, there are capabilities to read an Excel file (see the \emph{XLConnect}
package among others), but you can also just go into Excel and export
as a CSV file or the like and then read that into R. In general, it's
best not to pass around data files as Excel or other spreadsheet format
files because (1) Excel is proprietary, so someone may not have Excel
and the format is subject to change, (2) Excel imposes limits on the
number of rows, (3) one can easily manipulate text files such as CSV
using UNIX tools, but this is not possible with an Excel file, (4)
Excel files often have more than one sheet, graphs, macros, etc.,
so they're not a data storage format per se.
\end{enumerate}

\section{Reading data from text files into R}

\emph{read.table()} is probably the most commonly-used function for
reading in data, it reads in delimited files (\emph{read.csv()} and
\emph{read.delim()} are special cases of \emph{read.table()}). The
key arguments are the delimiter (the \emph{sep} argument) and whether
the file contains a header, a line with the variable names. We can
use \emph{read.fwf()} to read from a fixed width text file into a
data frame. 

The most difficult part of reading in such files can be dealing with
how R determines the classes of the fields that are read in. There
are a number of arguments to \emph{read.table()} and \emph{read.fwf()}
that allow the user to control the classes. One difficulty is that
character and numeric fields are sometimes read in as factors. Basically
\emph{read.table()} tries to read fields in as numeric and if it finds
non-numeric and non-NA values, it reads in as a factor. This can be
annoying.

Let's work through a couple examples. First let's look at the arguments
to \emph{read.table()}. Note that \emph{sep=''''} separates on any
amount of white space.

\begin{knitrout}
\definecolor{shadecolor}{rgb}{0.969, 0.969, 0.969}\color{fgcolor}\begin{kframe}
\begin{alltt}
\hlfunctioncall{getwd}()  \hlcomment{# a common error is not knowing what directory R is looking at}
\hlfunctioncall{setwd}(\hlstring{"../data"})
dat <- \hlfunctioncall{read.table}(\hlstring{"RTADataSub.csv"}, sep = \hlstring{","}, head = TRUE)
\hlfunctioncall{lapply}(dat, class)
\hlfunctioncall{levels}(dat[, 2])
dat2 <- \hlfunctioncall{read.table}(\hlstring{"RTADataSub.csv"}, sep = \hlstring{","}, head = TRUE, na.strings = \hlfunctioncall{c}(\hlstring{"NA"}, 
    \hlstring{"x"}), stringsAsFactors = FALSE)
\hlfunctioncall{unique}(dat2[, 2])
\hlcomment{# hmmm, what happened to the blank values this time?}
\hlfunctioncall{which}(dat[, 2] == \hlstring{""})
dat2[\hlfunctioncall{which}(dat[, 2] == \hlstring{""})[1], ]  # deconstruct it!
sequ <- \hlfunctioncall{read.table}(\hlstring{"hivSequ.csv"}, sep = \hlstring{","}, header = TRUE, colClasses = \hlfunctioncall{c}(\hlstring{"integer"}, 
    \hlstring{"integer"}, \hlstring{"character"}, \hlstring{"character"}, \hlstring{"numeric"}, \hlstring{"integer"}))
\hlcomment{# let's make sure the coercion worked - sometimes R is obstinant}
\hlfunctioncall{lapply}(sequ, class)
\hlcomment{# that made use of the fact that a data frame is a list}
\end{alltt}
\end{kframe}
\end{knitrout}


If possible, it's a good idea to look through the input file in UNIX
before reading into R to catch such issues in advance. Using less
on \emph{RTADataSub.csv} would have revealed these various issues,
but note that \emph{RTADataSub.csv} is a 1000-line subset of a much
larger file of data available from the kaggle.com website.

The basic function \emph{scan()} simply reads everything in, ignoring
lines, which works well and very quickly if you are reading in a numeric
vector or matrix. \emph{scan()} is also useful if your file is free
format - i.e., if it's not one line per observation, but just all
the data one value after another; in this case you can use \emph{scan()}
to read it in and then format the resulting character or numeric vector
as a matrix with as many columns as fields in the dataset. Remember
that the default is to fill the matrix by column.

If the file is not nicely arranged by field (e.g., if it has ragged
lines), we'll need to do some more work. \emph{readLines()} will read
in each line into a separate character vector, after which we can
process the lines using text manipulation. Here's an example from
some US meteorological data where I know from metadata (not provided
here) that the 4-11th values are an identifier, the 17-20th are the
year, the 22-23rd the month, etc.

\begin{knitrout}
\definecolor{shadecolor}{rgb}{0.969, 0.969, 0.969}\color{fgcolor}\begin{kframe}
\begin{alltt}
dat <- \hlfunctioncall{readLines}(\hlstring{"../data/precip.txt"})
id <- \hlfunctioncall{as.factor}(\hlfunctioncall{substring}(dat, 4, 11))
year <- \hlfunctioncall{substring}(dat, 17, 20)
year[1:5]
\end{alltt}
\begin{verbatim}
## [1] "I201" "I201" "I201" "I201" "I201"
\end{verbatim}
\begin{alltt}
\hlfunctioncall{class}(year)
\end{alltt}
\begin{verbatim}
## [1] "character"
\end{verbatim}
\begin{alltt}
year <- \hlfunctioncall{as.integer}(\hlfunctioncall{substring}(dat, 18, 21))
month <- \hlfunctioncall{as.integer}(\hlfunctioncall{substring}(dat, 22, 23))
nvalues <- \hlfunctioncall{as.integer}(\hlfunctioncall{substring}(dat, 28, 30))
\end{alltt}
\end{kframe}
\end{knitrout}


Note that for precip.txt, reading in using read.fwf() would be a good
strategy.

R allows you to read in not just from a file but from a more general
construct called a \emph{connection}. Here are some examples of connections:

\begin{knitrout}
\definecolor{shadecolor}{rgb}{0.969, 0.969, 0.969}\color{fgcolor}\begin{kframe}
\begin{alltt}
dat <- \hlfunctioncall{readLines}(\hlfunctioncall{pipe}(\hlstring{"ls -al"}))
dat <- \hlfunctioncall{read.table}(\hlfunctioncall{pipe}(\hlstring{"unzip dat.zip"}))
dat <- \hlfunctioncall{read.csv}(\hlfunctioncall{gzfile}(\hlstring{"dat.csv.gz"}))
dat <- \hlfunctioncall{readLines}(\hlstring{"http://www.stat.berkeley.edu/~paciorek/index.html"})
\end{alltt}
\end{kframe}
\end{knitrout}


If a file is large, we may want to read it in in chunks (of lines),
do some computations to reduce the size of things, and iterate. \emph{read.table()},
\emph{read.fwf()} and \emph{readLines()} all have the arguments that
let you read in a fixed number of lines. To read-on-the-fly in blocks,
we need to first establish the connection and then read from it sequentially. 

\begin{knitrout}
\definecolor{shadecolor}{rgb}{0.969, 0.969, 0.969}\color{fgcolor}\begin{kframe}
\begin{alltt}
con <- \hlfunctioncall{file}(\hlstring{"../data/precip.txt"}, \hlstring{"r"})
\hlfunctioncall{class}(con)
blockSize <- 1000  \hlcomment{# obviously this would be large in any real application}
nLines <- 3e+05
\hlfunctioncall{for} (i in 1:\hlfunctioncall{ceiling}(nLines/blockSize)) \{
    lines <- \hlfunctioncall{readLines}(con, n = blockSize)
\hlcomment{    # manipulate the lines and store the key stuff}
\}
\hlfunctioncall{close}(con)
\end{alltt}
\end{kframe}
\end{knitrout}


One cool trick that can come in handy is to create a \emph{text connection}.
This lets you 'read' from an R character vector as if it were a text
file and could be handy for processing text. For example, you could
then use \emph{read.fwf()} applied to \emph{con}.

\begin{knitrout}
\definecolor{shadecolor}{rgb}{0.969, 0.969, 0.969}\color{fgcolor}\begin{kframe}
\begin{alltt}
dat <- \hlfunctioncall{readLines}(\hlstring{"../data/precip.txt"})
con <- \hlfunctioncall{textConnection}(dat[1], \hlstring{"r"})
\hlfunctioncall{read.fwf}(con, \hlfunctioncall{c}(3, 8, 4, 2, 4, 2))
\end{alltt}
\begin{verbatim}
##    V1      V2   V3 V4   V5 V6
## 1 DLY 1000807 PRCP HI 2010  2
\end{verbatim}
\end{kframe}
\end{knitrout}


We can create connections for writing output too. Just make sure to
open the connection first.

Be careful with the directory separator in Windows files: you can
either do \emph{``C:\textbackslash{}\textbackslash{}mydir\textbackslash{}\textbackslash{}file.txt''}
or \emph{``C:/mydir/file.txt''}, but not \emph{``C:\textbackslash{}mydir\textbackslash{}file.txt''}.
{[}I think; I haven't checked this, so a Windows user should correct
me if I'm wrong.{]}


\section{Output from R}


\subsection{Writing output to files}

Functions for text output are generally analogous to those for input.
\emph{write.table()},\textbf{ }\emph{write.csv()}, and\textbf{ }\emph{writeLines()}
are analogs of \emph{read.table()}, \emph{read.csv()}, and \emph{readLines()}.
\emph{write()} can be used to write a matrix to a file, specifying
the number of columns desired. \emph{cat()} can be used when you want
fine control of the format of what is written out and allows for outputting
to a connection (e.g., a file).

And of course you can always save to an R data file using \emph{save.image()}
(to save all the objects in the workspace or \emph{save()} to save
only some objects. Happily this is platform-independent so can be
used to transfer R objects between different OS.


\subsection{Formatting output}

One thing to be aware of when writing out numerical data is how many
digits are included. For example, the default with \emph{write()}
and\textbf{\emph{ }}\emph{cat()} is the number of digits displayed
to the screen, controlled by \texttt{\emph{options()\$digits}}. (to
change this, do\texttt{\textbf{ }}\texttt{options(digits = 5)} or
specify as an argument to \emph{write()} or \emph{cat()}) If you want
finer control, use \emph{sprintf()}, e.g. to print out print out temperatures
as reals (``\emph{f}''=floating points) with four decimal places
and nine total character positions, followed by a C for Celsius:

\begin{knitrout}
\definecolor{shadecolor}{rgb}{0.969, 0.969, 0.969}\color{fgcolor}\begin{kframe}
\begin{alltt}
temps <- \hlfunctioncall{c}(12.5, 37.234324, 1342434324.79997, 2.3456e-06, 1e+10)
\hlfunctioncall{sprintf}(\hlstring{"%9.4f C"}, temps)
\end{alltt}
\begin{verbatim}
## [1] "  12.5000 C"        "  37.2343 C"        "1342434324.8000 C" 
## [4] "   0.0000 C"        "10000000000.0000 C"
\end{verbatim}
\end{kframe}
\end{knitrout}


\emph{cat()} is a good choice for printing a message to the screen,
often better than \emph{print()}, which is an object-oriented method.
You generally won't have control over how the output of a \emph{print()}
statement is actually printed.

\begin{knitrout}
\definecolor{shadecolor}{rgb}{0.969, 0.969, 0.969}\color{fgcolor}\begin{kframe}
\begin{alltt}
val <- 1.5
\hlfunctioncall{cat}(\hlstring{"My value is "}, val, \hlstring{".\textbackslash{}n"}, sep = \hlstring{""})
\end{alltt}
\begin{verbatim}
## My value is 1.5.
\end{verbatim}
\begin{alltt}
\hlfunctioncall{print}(\hlfunctioncall{paste}(\hlstring{"My value is "}, val, \hlstring{"."}, sep = \hlstring{""}))
\end{alltt}
\begin{verbatim}
## [1] "My value is 1.5."
\end{verbatim}
\end{kframe}
\end{knitrout}


We can do more to control formatting with \emph{cat()}:

\begin{knitrout}
\definecolor{shadecolor}{rgb}{0.969, 0.969, 0.969}\color{fgcolor}\begin{kframe}
\begin{alltt}
\hlcomment{# input}
x <- 7
n <- 5
\hlcomment{# display powers}
\hlfunctioncall{cat}(\hlstring{"Powers of"}, x, \hlstring{"\textbackslash{}n"})
\hlfunctioncall{cat}(\hlstring{"exponent   result\textbackslash{}n\textbackslash{}n"})
result <- 1
\hlfunctioncall{for} (i in 1:n) \{
    result <- result * x
    \hlfunctioncall{cat}(\hlfunctioncall{format}(i, width = 8), \hlfunctioncall{format}(result, width = 10), \hlstring{"\textbackslash{}n"}, sep = \hlstring{""})
\}
x <- 7
n <- 5
\hlcomment{# display powers}
\hlfunctioncall{cat}(\hlstring{"Powers of"}, x, \hlstring{"\textbackslash{}n"})
\hlfunctioncall{cat}(\hlstring{"exponent result\textbackslash{}n\textbackslash{}n"})
result <- 1
\hlfunctioncall{for} (i in 1:n) \{
    result <- result * x
    \hlfunctioncall{cat}(i, \hlstring{"\textbackslash{}t"}, result, \hlstring{"\textbackslash{}n"}, sep = \hlstring{""})
\}
\end{alltt}
\end{kframe}
\end{knitrout}



\section{File and string encodings}

Text (either in the form of a file with regular language in it or
a data file with fields of character strings) will often contain characters
that are not part of the {[}limited ASCII set of characters{]}(http://en.wikipedia.org/wiki/ASCII),
which has $2^{7}=128$ characters and control codes; basically what
you see on a standard US keyboard. So for non-ASCII files you may
need to deal with the text encoding (the mapping of individual characters
(including tabs, returns, etc.) to a set of numeric codes). There
are a variety of different encodings for text files, with different
ones common on different operating systems. UTF-8 is an encoding for
the Unicode characters that include more than 110,000 characters from
100 different alphabets/scripts. It's widely used on the web. Latin-1
encodes a small subset of Unicode and contains the characters used
in many European languages (e.g., letters with accents). 

The UNIX utility \emph{file}, e.g. \texttt{file tmp.txt} can help
provide some information. \emph{read.table()} in R takes arguments
\emph{fileEncoding} and \emph{encoding} that address this issue. The
UNIX utility \emph{iconv} and the R function \emph{iconv()} can help
with conversions.

In US installations of R, the default encoding is UTF-8; note that
various types of information are interpreted in US English with the
encoding UTF-8:

\begin{knitrout}
\definecolor{shadecolor}{rgb}{0.969, 0.969, 0.969}\color{fgcolor}\begin{kframe}
\begin{alltt}
\hlfunctioncall{Sys.getlocale}()
\end{alltt}
\begin{verbatim}
## [1] "LC_CTYPE=en_US.UTF-8;LC_NUMERIC=C;LC_TIME=en_US.UTF-8;LC_COLLATE=en_US.UTF-8;LC_MONETARY=en_US.UTF-8;LC_MESSAGES=en_US.UTF-8;LC_PAPER=C;LC_NAME=C;LC_ADDRESS=C;LC_TELEPHONE=C;LC_MEASUREMENT=en_US.UTF-8;LC_IDENTIFICATION=C"
\end{verbatim}
\end{kframe}
\end{knitrout}


With strings already in R, you can convert between encodings with
\emph{iconv()}:

\begin{knitrout}
\definecolor{shadecolor}{rgb}{0.969, 0.969, 0.969}\color{fgcolor}\begin{kframe}
\begin{alltt}
text <- \hlstring{"_Melhore sua seguran\textbackslash{}xe7a_"}
\hlfunctioncall{iconv}(text, from = \hlstring{"latin1"}, to = \hlstring{"UTF-8"})
\end{alltt}
\begin{verbatim}
## [1] "_Melhore sua seguran�a_"
\end{verbatim}
\begin{alltt}
\hlfunctioncall{iconv}(text, from = \hlstring{"latin1"}, to = \hlstring{"ASCII"}, sub = \hlstring{"???"})
\end{alltt}
\begin{verbatim}
## [1] "_Melhore sua seguran???a_"
\end{verbatim}
\end{kframe}
\end{knitrout}


You can also mark a string with an encoding, so R knows how to display
it correctly:

\begin{knitrout}
\definecolor{shadecolor}{rgb}{0.969, 0.969, 0.969}\color{fgcolor}\begin{kframe}
\begin{alltt}
x <- \hlstring{"fa\textbackslash{}xE7ile"}
\hlfunctioncall{Encoding}(x) <- \hlstring{"latin1"}
x
\end{alltt}
\begin{verbatim}
## [1] "fa�ile"
\end{verbatim}
\begin{alltt}
\hlcomment{# playing around...}
x <- \hlstring{"\textbackslash{}xa1 \textbackslash{}xa2 \textbackslash{}xa3 \textbackslash{}xf1 \textbackslash{}xf2"}
\hlfunctioncall{Encoding}(x) <- \hlstring{"latin1"}
x
\end{alltt}
\begin{verbatim}
## [1] "� � � � �"
\end{verbatim}
\end{kframe}
\end{knitrout}


An R error message with \textquotedbl{}multi-byte string\textquotedbl{}
in the message often indicates an encoding issue. In particular errors
often arise when trying to do string manipulations in R on character
vectors for which the encoding is not properly set. Here's an example
with some Internet logging data that we used last year in 243 in a
problem set and which caused some problems.

\begin{knitrout}
\definecolor{shadecolor}{rgb}{0.969, 0.969, 0.969}\color{fgcolor}\begin{kframe}
\begin{alltt}
\hlfunctioncall{load}(\hlstring{"../data/IPs.RData"})  # loads in an object named \hlstring{'text'}
tmp <- \hlfunctioncall{substring}(text, 1, 15)
\end{alltt}


{\ttfamily\noindent\bfseries\color{errorcolor}{\#\# Error: invalid multibyte string at '<bf>a7lw8'}}\begin{alltt}
\hlcomment{# the issue occurs with the 6402th element (found by trial and error):}
tmp <- \hlfunctioncall{substring}(text[1:6401], 1, 15)
tmp <- \hlfunctioncall{substring}(text[1:6402], 1, 15)
\end{alltt}


{\ttfamily\noindent\bfseries\color{errorcolor}{\#\# Error: invalid multibyte string at '<bf>a7lw8'}}\begin{alltt}
text[6402]  \hlcomment{# note the Latin-1 character}
\end{alltt}
\begin{verbatim}
## [1] "from 5#c\xbfa7lw8lz2nX,%@ [128.32.244.179] by ncpc-email with ESMTP\n(SMTPD32-7.04) id A06E24A0116; Mon, 10 Jun 2002 11:43:42 +0800"
\end{verbatim}
\begin{alltt}
text <- \hlfunctioncall{iconv}(text, from = \hlstring{"latin1"}, to = \hlstring{"UTF-8"})
text[6402]
\end{alltt}
\begin{verbatim}
## [1] "from 5#c�a7lw8lz2nX,%@ [128.32.244.179] by ncpc-email with ESMTP\n(SMTPD32-7.04) id A06E24A0116; Mon, 10 Jun 2002 11:43:42 +0800"
\end{verbatim}
\begin{alltt}
tmp <- \hlfunctioncall{substring}(text, 1, 15)
tmp[6402]
\end{alltt}
\begin{verbatim}
## [1] "from 5#c�a7lw8l"
\end{verbatim}
\begin{alltt}
\hlcomment{# Interesting:}
\hlfunctioncall{table}(\hlfunctioncall{Encoding}(text))
\end{alltt}
\begin{verbatim}
## 
## unknown   UTF-8 
##    6932       4
\end{verbatim}
\end{kframe}
\end{knitrout}



\end{document}
